
% JuliaCon proceedings template
\documentclass{juliacon}
\setcounter{page}{1}
\usepackage{amsmath}
%\usepackage{tikz}
%\usepackage{tikz-qtree}
\usepackage{booktabs}
%\usepackage{svg}

%\usetikzlibrary{graphs,graphdrawing}
%\usegdlibrary{force}

\graphicspath{{assets/}}

\begin{document}

% **************GENERATED FILE, DO NOT EDIT**************

\title{\texttt{EinExprs}: Contraction Paths as Symbolic Expressions}

\author[1]{Sergio Sanchez-Ramirez}
\author[1]{Jofre Vallès-Muns}
\author[1, 2]{Artur Garcia-Saez}
\affil[1]{Barcelona Supercomputing Center, 08034 Barcelona, Spain}
\affil[2]{Qilimanjaro Quantum Tech., 08014 Barcelona, Spain}

\keywords{Julia, Tensor Networks, Contraction Path, Symbolic Expressions, Optimization}



\maketitle

\begin{abstract}

Tensor Networks are graph representations of summation expressions in which vertices represent tensors and edges represent tensor indices or vector spaces.
In this work, we present \texttt{EinExprs.jl}, a Julia package for contraction path optimization that offers state-of-art optimizers.
We propose a representation of the contraction path of a Tensor Network based on symbolic expressions.
Using this package the user may choose among a collection of different methods such as Greedy algorithms, Hypergraph partitioning.
%We benchmark this library with examples obtained from the simulation of Random Quantum Circuits (RQC), a well known example where Tensor Networks provide state-of-the-art methods.
We benchmark this library against ... with random tensor networks.

\end{abstract}

\section{Introduction}
A Tensor Network is a collection of tensors connected by common indices indicating contraction operations. Despite being extensively used in different fields of Physics such as Quantum Information \cite{evenbly2022practical} or Condensed Matter\cite{SCHOLLWOCK201196}, recently they have received much attention due to their capabilities to simulate Quantum circuits, a task hard even for supercomputers \cite{PhysRevX.10.041038}. The necessity to improve Tensor Network methods emerges from the computational resources required to manipulate these structures. Currently, Tensor Network methods are state of the art for quantum circuit simulation.

Tensor Networks are equivalent to a graph representation of Einstein summation expressions (a.k.a. \textit{einsum}) in which vertices represent tensors and edges represent tensor indices or vector spaces. A tensor $T$ is encoded by the Tensor Network, and can be exactly computed by contracting the tensors following the summation operations. As an example, using Einstein summation rules for common indices, $T$ is the result of the contraction of several Tensors:

\begin{equation}\label{eq:tn-example}
    T = A_{im} B_{ijp} C_{jkn} D_{klp} E_{mno} F_{lo} 
\end{equation} 

Any \textit{einsum} expression can be reinterpreted diagrammatically using Tensor Networks. For example, Equation~\ref{eq:tn-example} is represented graphically as shown in Figure~\ref{fig:tn-example}. The order in which the tensors are contracted highly affects the computational cost of the simulation. Indeed, exact tensor network contraction is a \#P-complete problem~\cite{garcia2012exact}. Finding the optimal contraction path of a tensor network is known to be linked to the optimal tree decomposition problem of the underlying graph \cite{markov2008simulating}. This is equivalent to finding the treewidth of such graph, which is a well-known NP-complete problem.

\begin{figure}[h]
    \centering
    %\begin{tikzpicture}[baseline]
    %    \tikz [
    %        spring electrical layout,
    %        node distance=15mm,
    %        iterations=100,
    %        xshift=4.5cm,
    %    ] {
    %        \foreach \i in {A,B,C,D,E,F}
    %            \node (\i) [draw,fill=orange!20,circle,minimum size=0.7cm] {$\i$};
        
    %        \draw (A) edge node[above,sloped] {$i$} (B)
    %                  edge node[above,sloped] {$m$} (E)
    %              (B) edge node[above,sloped] {$j$} (C)
    %                  edge node[above,sloped] {$p$} (D)
    %              (C) edge node[above,sloped] {$n$} (E)
    %                  edge node[above,sloped] {$k$} (D)
    %              (D) edge node[above,sloped] {$l$} (F)
    %              (E) edge node[above,sloped] {$o$} (F);
    %    };
    %\end{tikzpicture}
    \includegraphics[width=\columnwidth]{tn-example.png}
    \caption{Diagrammatic representation of Equation~\ref{eq:tn-example} as a Tensor Network.}
    \label{fig:tn-example}
\end{figure}


In this work we present \texttt{EinExprs.jl}, a package forTensor Network contraction path optimization and visualization. Many of the Tensor Networks found in the literature have some kind of structure. As with many NP-complete problems, this structure can be exploited to reduce the complexity of the problem. Recent developments in the field have demonstrated that some heuristics are well-suited for finding quasi-optimal contraction paths. \texttt{EinExprs.jl} aims to be the reference package for the development of new algorithms by providing an easy interface along the fastest implementations of well-known algorithms.
